\label{sec:related}
\subsection{Arch}
Arch Linux \cite{wiki:arch} is a lightweight GNU/Linux distribution.
It was created by the Canadian Judd Vinet between 2001 and 2002 and further developed until today.
The renowned Arch Wiki was created in 2005. Since 2007 Aaron Griffin leads the development, which is completely done by part time volunteers in order to remain free from corporate influence.

Arch Linux follows the rolling-release model instead of periodical releases to provide the users the newest possible stable version of every program.
Upstream software is modified as little as possible and only as much as needed to integrate with the rest of the distribution.

Arch makes many features available earlier than other distributions, such as the systemd init system, software raids and new file systems.
Systemd \cite{wiki:systemd} replaced the \emph{System V init system} in 2013 \footnote{\url{https://bbs.archlinux.org/viewtopic.php?pid=1149530\#p1149530} accessed June 2, 2017}.
It provides a system- and service manager which starts the rest of the system and manages daemons and background services.
Systemd services are easy to handle and to configure.

Arch provides open-source packages as well as proprietary software to let the user decide if he wants functionality or if he wants to stay with the open-source ideology when they conflict.
Users can install them with the lightweight package manager \texttt{pacman}, which is one of the core tools of Arch.
Pacman uses a simple binary package format and an easy-to-use build system. It allows to manage the packages from the official repositories as well personal builds.

Arch is a lightweight distribution, which means that the user has to install and configure everything himself.
Arch does not want to be user-friendly, but rather user-centric.
This is achieved by providing the user with many possibilities to create his own personalized system and help him fix issues himself.
Arch encourages tinkering and caters to power users who want to understand and configure much of their system.
There are mailing-lists, an official Wiki, an official Forum, a bug tracker and several other places where users can get help.
Most members of the community are experienced Linux users who are willing to read documentations and solve problems on their own with the help of the community, as well as help others.

It is notoriously hard to estimate the popularity of Linux distributions, and so it is hard to tell how wide-spread Arch Linux is.
Arch users can voluntarily submit a list of their installed packages for statistical evaluation by installing and calling a script called \texttt{pkgstats}.
These stats show 13,353 different IP addresses from Arch users as of June 12, 2017~\cite{arch:stats}, which is fairly small.
However, no one knows how to control for administrators using many Arch machines behind one router, which seems to be common, and for the (probably many) users who do not submit these statistics.
According to the results of the yearly Reddit Linux Distribution Survey on \texttt{/r/linux} in 2015, 28.47\% of the participating users used Arch Linux~\cite{linuxsurvey}, with only Ubuntu appearing in the same order of magnitude.
Overall, it seems that Arch is a small, but very active distribution with comparatively vocal users. It's popularity has been steadily increasing for the past few years.

\subsubsection{AUR}
The Arch User Repository (AUR) is an unofficial, community-driven repository for Arch users.
It contains \texttt{PKGBUILD}s which are package descriptions containing all information needed to create packages from source using a tool called \texttt{makepkg}.
These packages can then be installed with \texttt{pacman}, Arches normal package manager.
Everyone can upload packages to the AUR as well as download them.
The uploaded packages are not verified or audited in any way in order to minimize the delay between the upload and the availability of a version.
Users can vote for packages; Packages with enough votes may get adopted by a maintainer and moved to the official Arch community repository.
If all maintainers of a package no longer want to maintain it, it can be orphaned. Any AUR user may adopt orphan packages.
\cite{wiki:AUR}

\subsection{Blockchain}\label{sec:ABCblockchain}
A blockchain is similar to a distributed database system which is not owned by a single user.
The first blockchain was implemented by Santoshi Nakamoto in 2008.
This blockchain is based on research of the early 1990s where the concept of cryptographically secured blocks was first described.
It is also the core of the bitcoin currency.
Since then several blockchain approaches where deployed.

As opposed to classic databases, blockchains maintain the entire transaction history.
Every user can look into this history and can track other users because the transactions are not anonymous.
Because of that, blockchains are the first digital approach of decentral database solving the double-spending problem.

Normally blockchain networks are peer-to-peer.
If a user wants to add a transaction to the blockchain, the transaction is encrypted, sent to all users, and verified.
If the majority of users validates the transaction, the data is added to the blockchain in the next block.
Underwood describes these transactions as ``... secure, trusted, auditable, and immutable''~\cite{Underwood}, which means that the transactions cannot be manipulated without a majority consent.

Users do not trust the result of a single node, but that of the majority.
As a result, blockchains are highly ``Byzantine''-fault tolerant.
This means that attackers need more than 50\% of the computation power to manipulate a blockchain (because then the attacker can produce more blocks than the rest).
If attackers gets more than 50\% of the computing power, they can manipulate the chain because the longest chain is always considered the canonical one.
Shorter tails become orphans and are rejected by the network.

Because blockchains are distributed and every user has a synchronized copy of the chain (which often requires several GB of disk space), they do not have a single point of failure or require backups.
The benefits of a blockchain compared to a conventional database are: (a) The blockchain can be directly shared because transactions contain their own proof as provided by a sophisticated cryptographic algorithm. (b) The blockchain is more robust (even than distributed databases).
Blockchains are often used to store transactions of concurrency (e.g. bitcoin), identity management or medical records.
\cite{Underwood,blockchain:eco}\newline\newline

To create new blocks \textbf{miners} are needed (proof of work).
A miner is the computer which has found a hash since the last block based on the transaction information and other data such as the hash of the last block.
This hash has to fulfill several requirements; The more there are, the more the mining difficulty increases.
It is hard to find a hash but it is easy to check if the hash fulfills all the requirements.
This is important because it makes it possible for every node to verify every block.

The mining difficulty describes the time effort required for mining of a block.
If the mining difficulty is high, the miner receives more \emph{coins} (Ethereum uses \emph{ether}) to reward the effort.
Coins are needed to perform transactions.
However, the effort and the payment have to be in sound relation to one another: that is, the miner's effort must not deceed or exceed the payment.
If the mining difficulty is too high in relation to the reward, it can lead to less blocks per hour which by implication leads to high transaction latencies because the transactions are not validated for considerable time.
If the opposite happens, there will be a low transaction per block ratio which makes the blockchain unnecessarily big; Users could also flood the chain with very many transactions, weakening the meaning of majority consent.\newline\newline

To provide functionality extending the transaction of coins, the blockchain has to accept \textbf{smart contracts}.
Smart contracts use computerized transaction protocols to execute the terms of a contract which are agreed upon by the users.
The term \emph{Smart Contract} might be misleading, as in reality it is a sequence of handling arrangements which map the contract clauses into code \cite{ecolex} --- they are effectively \emph{functions} of which the semantics may not be changed after all parties agreed on them.
Smart Contracts defining rules, penalties around an agreement and automatically check and impose them.

Without smart contracts, a blockchain can only handle the transaction of coins.
With smart contracts, a blockchain can be used as a database, containing objects which can be created, modified and deleted by \enquote{intelligent} transactions.
Additionally, object properties may be read without performing a transaction, allowing users to read information without paying currency.
The functions requiring coins are executed by every miner and are ACID transactions (atomic, consistent, isolated, durable).
Since the contract code is included in the chain, it is guaranteed to be immutable and therefore impossible to manipulate.
This guarantees that one is able to run secure code on the blockchain.

By avoiding middleman (e.g. lawyers), smart contracts are safer and often cheaper than their real-world counterpart.
They are safer because all parties communicate directly, and cheaper because the transaction fee is normally cheaper than the costs of a lawyer or notary.
Smart contracts can be used in different systems:
The Governments could use them for electronic voting systems, Companies already use them (e.g. Ethereum \footnote{\url{https://entethalliance.org/}) accessed on June 20, 2017}) in different ways and they are also used in healthcare systems (e.g. gem \footnote{\url{https://gem.co/} accessed on June 20, 2017}) which store the data in the blockchain and make it available for authorized users (e.g. doctors).
The healthcare-example also shows another advantage of smart contracts.
People cannot ``lose'' their data because its in the blockchain which is distributed.
Also Smart Contracts are more accurate than manually filled forms because they avoid manual errors.

Blockgeeks explains Smart Contracts with the following example:

\begin{displayquote}
	Suppose you rent an apartment from me. You can do this through the blockchain by paying in cryptocurrency. You get a receipt which is held in our virtual contract; I give you the digital entry key which comes to you by a specified date. If the key doesn’t come on time, the blockchain releases a refund. If I send the key before the rental date, the function holds it releasing both the fee and key to you and me respectively when the date arrives. The system works on the If-Then premise and is witnessed by hundreds of people, so you can expect a faultless delivery. If I give you the key, I’m sure to be paid. If you send a certain amount in bitcoins, you receive the key. The document is automatically canceled after the time, and the code cannot be interfered by either of us without the other knowing, since all participants are simultaneously alerted.~\cite{blockgeeks}
\end{displayquote}

Finally, one must mention that smart contracts are not perfect. Bugs in the code can lead to vulnerabilities which can be used to attack a blockchain.
The most popular example for this case is the ethereum-hard-split in 2016 \cite{heise-dao} where a bug in a smart contract was used to attack the Ethereum blockchain and steal 3,6 million ether-coins.
Furthermore, misunderstandings in traditional contracts can be fought over in court, but smart contracts cannot be modified or prevented from executing.
\cite{blockgeeks}


\subsection{Ethereum}
Ethereum is a blockchain-based, open-source, distributed computing platform developed by Vitalik Buterin, Gavin Wood and Jeffrey Wilcke.
It provides smart contract functionality. These smart contracts are usually written in the specially developed, Turing-complete language \emph{Solitidy}.
Solitidys syntax is similar to the syntax of JavaScript.
Every node runs an instance of the \emph{Ethereum Virtual Machine} (EVM), which can execute code of arbitrary complexity; The smart contracts are executed on the EVM.
Every EVM in the network executes the same instructions.

Ethereum rewards mining effort with it's currency, ether-coins.
As of June 3, 2017, the blockchains holds 92,2 Million Coins and is worth more than 20 billion US\$, which makes it the second most valuable blockchain after Bitcoin.
While every transaction in bitcoin costs exactly the same, transactions in ethereum have different fees based on the required storage, code complexity and bandwidth usage of smart contracts they interact with.
Beside it's main blockchain, Ethereum provides all the required infrastructure to create private blockchains.
\cite{ethereum, wiki:ethereum}

Bloomberg \cite{bloomberg:eth} describes Ethereum as \enquote{shared software that can be used by all but is tamperproof. You can safely do business with someone you don’t know, because terms are spelled out in a \enquote{smart contract} embedded in the blockchain. There could be blockchain versions of Uber and Airbnb that are peer-to-peer: No company would need to sit in the middle of the transaction to gather data about your spending habits or collect a fee.} %\cite{bloomberg:eth}


% Ethereum - Solidity
% RPC
% Bash as a programming language
% Python - Urwid

\subsubsection*{Ethereum --- Solidity}
Programming on a blockchain is a very interesting concept, but it seemed like it would take some getting used to. Thankfully, Solidity, the main language used on Ethereum, is a cleanly designed language which abstracts the blockchain away very nicely.

In practice, writing for Ethereum turned out to be enjoyable. Solidity reads like a half-way mix between C and Javascript, and most things that we rely on in this project happen automatically: Guaranteeing that the code cannot be modified, the ACID properties of transactions, etc. Solidity even supports some automatic formal verification, but not yet for \texttt{struct}s, so we cannot make use of it for our program.

All in all, Solidity is ideally suited for writing secure, \enquote{intelligent} (self-enforcing) databases.

\subsubsection*{JSON RPC}
The JSON PRC is used by \texttt{aursec-chain} for remote access to the block. Methods can be called or transactions can be sent by sending a JSON Object to the block chain. The answer from the block chain is also always a JSON Object. We learned how to parse JSON using python- and bash scripts. The best example is visible in the code of aursec-tui (Section~\ref{sec:tui}).

\subsubsection*{Bash}
Using Bash as a programming language was interesting. The language has many pitfalls~\cite{bash:pitfalls} and its syntax can be strange, even arcane; At the same time, we were often surprised by the advanced features that are provided directly by the language, such as string substitutions, regular expression matching or associative arrays.

These are the lessons that we learned about Bash (and whish we had known from the start):

\begin{itemize}
	\item Treat it as a functional programming language while structuring a program --- design in pipelines/streams.
	\item Error handling is hard to do well.
	Whenever possible, use \texttt{set -e} to abort on error and exit handlers for cleanup.
	\item Use a static analysis tool such as \emph{Shellcheck}~\cite{bash:shellcheck} to navigate the syntactic pitfalls.
	\item Find or invent a good way to test.
\end{itemize}

Bash doesn't provide or recommend a canonical testing framework, but associative arrays and \texttt{eval} allowed us to write our own system for basic unit tests with named test cases, commands to execute, and expected invariants.
We used it to great effect in narrowing down the best \texttt{firejail} sandbox ruleset, e.g. preventing \texttt{makepkg} from writing to folders other than \texttt{pwd}.

Overall, Bash is powerful and probably the perfect tool for this job, but so different from common programming languages that it took some getting used to.

\subsubsection*{Python-Urwid}
Urwid \cite{urwid} is a \emph{Terminal User Interface} (TUI) library for Python. It provides ready-made widgets which make it easy to create structured user interfaces. In total there are 20 different Widget-Classes which differ in type (box/flow/fixed) and category (basic/graphic/decoration/container). The basic and the graphic widgets contain the content. The decoration and container widgets are used to structure the TUI.

Buttons, check-boxes and key bindings allow to call functions. This function may add, remove or change widgets on runtime.

Using this library, we learned how to write simple TUIs which are completely modular and can be extended anytime.

The Arch Linux User Repository is fundamentally insecure.
It was designed for simplicity and rapid iteration, leaving its users unprotected against many kinds of attacks.

This project increases its security by:
\begin{enumerate}
	\item Analyzing its security issues
	\item Creating a threat model
	\item Solving one of these issues, the \emph{lack of verified release checksums}, thereby protecting users against targeted attacks and compromised sources.
\end{enumerate}

For this purpose we compare the downloaded package sources with the hashes submitted by other users in a secure database where every user is allowed to upload each hash only once.
The database of choice was a blockchain capable of smart contracts because it is distributed and fulfills our security requirements.

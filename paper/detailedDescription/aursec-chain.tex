\subsubsection{API - aursec-chain}\label{sec:aursec-chain}
\texttt{aursec-chain} is a shell script which allows the user and other scripts to interact with our blockchain.
The script itself communicates with the blockchain through the JSON RPC.
It provides the commands to start and stop mining, mine a number of blocks, get the current consensus hash of a versioned package and commit a hash of a versioned package.
In our command pipeline, \texttt{aursec-verify-hashes} calls \texttt{aursec-chain get-hash \$ID}.
For our example \texttt{PKGBUILD}, the command looks like this:

\begin{verbatim}
aursec-chain get-hash aursec-git-v0.11.r0.15ea0e9-1
\end{verbatim}

this returns

\begin{verbatim}
c54d99fb6c4797cd994fcef74082396aaadbbcf44021932e04c65b77745e45cd 2
\end{verbatim}

which means that \texttt{\ldots745e45cd} is the current consensus hash and it was committed two times.

Furthermore \texttt{aursec} calls \texttt{aursec-chain commit-hash \$ID \$HASH} when the local hash is trusted in order to include it in the blockchain and update the consensus.

For our example \texttt{PKGBUILD}, the command looks like this:

\begin{verbatim}
aursec-chain commit-hash aursec-git-v0.11.r0.15ea0e9-1
		0bccd006ecf636e91cc924a7e6f9f25c26164de715c8972d146505b8e7ec1afc
\end{verbatim}

\texttt{aursec-chain} is also indirectly used by the systemd-service \texttt{aursec-blockchain-mine.timer}, which periodically mines blocks.

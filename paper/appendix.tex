\section{A sample \texttt{PKGBUILD}}\label{sec:our-pkgbuild}

\begin{minted}[linenos]{bash}
# Maintainer: Bennett Piater <bennett at piater dot name>

pkgname=(aursec-git aursec-tui-git)
pkgver=v0.11.r0.15ea0e9
pkgrel=1
pkgdesc='Verify AUR package sources against hashes stored in a blockchain.'
arch=(any)
url="https://github.com/clawoflight/${pkgbase%-git}"
license=('custom:MPL2')

provides=("${pkgname%-git}")
conflicts=("${pkgname%-git}")

depends=(firejail geth vim bc)
makedepends=(pandoc git)
checkdepends=(shellcheck)

source=("git+https://github.com/clawoflight/${pkgname%-git}.git")
sha256sums=('SKIP')
validpgpkeys=('871F10477DB3DDED5FC447B226C7E577EF967808')

pkgver() {
    cd "${pkgname%-git}"
    printf "%s" "$(git describe --long | sed 's/\([^-]*-\)g/r\1/;s/-/./g')"
}

build() {
    cd "${pkgname%-git}/aursec"
    make
    cd "../tui"
    make
}

check() {
    cd "${pkgname%-git}/aursec"
    make -k check
}

package_aursec-git() {
    install=aursec-git.install
    optdepends=("aursec-tui: to manually inspect the blockchain.")

    cd "${pkgname%-git}/aursec"
    make PREFIX="/usr" DESTDIR="$pkgdir/" install
}

package_aursec-tui-git() {
    pkgdesc='Inspect the aursec blockchain'
    depends=(python python-requests python-urwid aursec)
    provides=(aursec-tui)
    conflicts=(aursec-tui)

    cd "aursec/tui"
    make PREFIX="/usr" DESTDIR="$pkgdir/" install
}
\end{minted}

\section{Smart Contract}\label{sec:contract}
\begin{minted}[linenos]{javascript}
// This Source Code Form is subject to the terms of the Mozilla Public
// License, v. 2.0. If a copy of the MPL was not distributed with this
// file, You can obtain one at http://mozilla.org/MPL/2.0/.
// Copyright 2016-2017 Lukas Krismer and Bennett Piater.

pragma solidity ^0.4.0;
import "contracts/Owned.sol";

/**
 * @title Registry of AUR package hashes
 * @author Bennett Piater
 */
contract AURPackageRegistry is Owned {

    // Struct that holds the consensus
    // and the number of submissions for each hash for a package.
    struct PackageData {
        string currentConsensusPkgHash;
        mapping (string => uint) timesHashSubmitted;
    }

    // Map package IDs "$pkgname-$pkgver-$pkgrel" to the corresponding struct
    mapping (string => PackageData) packages;

    // Map a hash to a map of blockchain wallet addresses (users).
    // This allows checking whether a user submitted a hash.
    mapping (string => mapping (address => bool)) addressesThatSubmittedAHash;

    event PkgHashSubmitted(string indexed packageID, string pkgHash,
        uint submissionCount, address indexed submitter);
    event ConsensusFormed(string indexed packageID, string pkgHash,
        uint submissionCount);

    /**
     * @notice Get the current consensus and how many nodes submitted it
     * for a given package,version,release combination.
     *
     * @param packageID The id of the package to submit: pkgname-pkgver-pkgrel
     *
     * @return pkgHash The hash of the package, or the empty string if none is stored.
     * @return submissionCount The number of nodes that submitted this hash
     */
    function getCurrentConsensus(string packageID) constant
    returns (string pkgHash, uint submissionCount) {
        string hash = packages[packageID].currentConsensusPkgHash;
        return (hash, packages[packageID].timesHashSubmitted[hash]);
    }

    /**
     * @notice Submit a new hash for a package.
     *
     * @param packageID The id of the package to submit: pkgname-pkgver-pkgrel
     * @param pkgHash The hash of the package
     * @return success Whether the submission succeeded
     */
    function submitPkgHash(string packageID, string pkgHash) returns (bool success) {
        // Only allow every address (=user) to submit a hash once
        if (addressesThatSubmittedAHash[pkgHash][msg.sender])
            return false;
        addressesThatSubmittedAHash[pkgHash][msg.sender] = true;

        PackageData package = packages[packageID];
        package.timesHashSubmitted[pkgHash] += 1;
        // Trigger notification
        PkgHashSubmitted(packageID, pkgHash,
            package.timesHashSubmitted[pkgHash], msg.sender);

        // If this hash has become the new consensus
        if (package.timesHashSubmitted[pkgHash] >
            package.timesHashSubmitted[package.currentConsensusPkgHash]) {

            package.currentConsensusPkgHash = pkgHash;
            // Trigger notification
            ConsensusFormed(packageID, pkgHash, package.timesHashSubmitted[pkgHash]);
        }

        return true;
    }
}
\end{minted}

\newpage
\section{Adding Bootnodes}\label{sec:init-boot}
This section explains how mining and non-mining bootnodes can be created and integrated into our network.\newline

\textbf{Mining bootnodes} can only be initialized automatically with \emph{aursec-init} on Arch Linux.
After doing so, instead of starting the systemd services, use the command
\begin{verbatim}
sudo su aursec -c '/usr/bin/geth --fast --port 30200 --rpcport 8105 --nodiscover
--ipcdisable --rpc --rpcapi "eth,web3,miner" --datadir /var/aursec/chain --cache 512
--networkid 42 --verbosity 4 --unlock 0 --password /var/aursec/password console'
\end{verbatim}

to get a geth console. To get your node-url run \texttt{admin.nodeInfo}. Copy the url and close the console (Ctrl+D or exit). To start the node use

\begin{verbatim}
sudo su aursec -c '/usr/bin/geth --fast --port 30200 --rpcport 8105 --nodiscover
--ipcdisable --rpc --rpcapi "eth,web3,miner" --datadir /var/aursec/chain --cache 512
--networkid 42 --verbosity 4 --unlock 0 --password /var/aursec/password'
\end{verbatim}

Now start the \texttt{aursec-blockchain-mine.timer}.
To integrate the bootnode to the network, read the corresponding paragraph below.\linebreak

\textbf{Non-mining bootnodes} only require the \texttt{geth} package and can be set up on any system.
To initialize the bootnode run the following commands:
\begin{verbatim}
> bootnode -genkey $PathToKeyfile
> bootnode -nodekey $PathToKeyfile -addr 30200
\end{verbatim}
The node-url will be shown in the first line after the second command.
To integrate the bootnode into the network, read the following paragraph.


\paragraph*{Integrate the Bootnode}
Copy the node-url consisting of \texttt{nodeUrl@yourip:30200} to our \texttt{aursec-blockchain.service} \footnote{https://github.com/clawoflight/aursec/blob/master/aursec/lib/aursec-blockchain.service} after the \texttt{--bootnodes} argument and submit a pull request.


\newpage
\section{Contributions}\label{sec:resp}
We split our work into areas of responsibility for which one of us was responsible (see Table \ref{tab:responsibilities}). However, usually both of us worked on any given area.
\begin{table}[!htb]
\centering
\caption{Contributions}
\label{tab:responsibilities}
\begin{tabular}{|l|L{5.5cm}|L{5.5cm}|}
\hline
Name & Code &  Paper \\ \hline
Bennett Piater & Solidity (Smart Contracts), automatic mining, hashing of packages, main command pipeline , aurutils wrapper, AUR-Packages   &
Abstract,
Motivation (Complete Section \ref{sec:motivation}),
AURsec (Main tool up to and including architecture, Systemd services, Integration with AUR Helpers ) ( \Cref{sec:aursec,sec:aur-architecture,sec:aursec-systemd,sec:integration} including Appendix \ref{sec:our-pkgbuild}),
Solidity code (Appendix \ref{sec:contract}),
Evaluation (Complete Section \ref{sec:eval}),
Discussion (Complete Section \ref{sec:discuss})
\\ \hline
Lukas Krismer &  Initialization of the blockchain, Integration of the contract, bash-API for ethereum , Terminal-User-Interface, Initialization of the bootnode &
Forword,
Introduction (Complete Section \ref{sec:intro}),
Related Work (Complete Section \ref{sec:related}),
AURsec(workflow, blockchain implementation, initialization, API, Terminal User Interface, bootnode) ( \Cref{sec:AURsec,sec:workflow,sec:aur-block,sec:aursec-init,sec:aursec-chain,sec:tui} including Appendix \ref{sec:init-boot}),
Contributions (Appendix \ref{sec:resp})\\ \hline
\end{tabular}
\end{table}

